\documentclass{article}
\usepackage[utf8]{inputenc}
\usepackage{hyperref}

\title{Portfolio}
\author{Dante Niewenhuis}

\begin{document}

\maketitle

\section*{Week 1}

\subsection*{Reading}
\begin{itemize}
    \item[] \textbf{R.1: } \textit{Read Kahneman part 1}
    \item[] Following is a small summary of the text. 
    Notes that I have taken during reading can be found on google Drive in the Notes folder.
    \item[] \textbf{\textit{Summary:}}  
    \item[] \textbf{R.2}
    \item[] Following is a small summary for each of the papers read. 
    Notes that I have taken during reading can be found on google Drive in the Notes folder.
    \item[] 
    \begin{itemize}
        \item[] \textit{Software Requirements and Specifications: }
        \item[] \textit{Contextual Design: }
        \item[] \textit{Setting the Scene: }
        \item[] \textit{TaskModeling: }   
    \end{itemize} 
    \item[] \textbf{R.3}
    \item[] TODO
    \item[] \textbf{R.4}
    \item[] TODO
\end{itemize}

\subsection*{Exercises}
\begin{itemize} 
    \item[] \textbf{E.1}
    \item[] TODO
    \item[] \textbf{E.2}
    \item[] My favourite interviewer is Louis Theroux
    \item[] \textbf{E.3}
    \item[] The recordings of my interviews can be found on Drive in the interview folder. 
    Following are what I found to be the most interesting thing from the interviews.
    After this I will reflect on the interviews.
    \item[] 
    \begin{itemize}
        \item[] \textbf{Clement Julia: }  
        Clement does not have any experiences with online learning so far. 
        This made it more difficult for us to talk specifically about online learning. 
        Instead we discussed what he finds important for courses in general. 
        Clement stated that structure in lectures is very important. 
        An example he gave to reach this is by having a clear first slide which shows 
        what will be discussed in the lecture and how this is connected to the rest of the course. 
        \item[] \textbf{Lucas Steehouwer: } 
        Lucas has experience with online while he was doing his previous master last year. 
        For Lucas it is important that courses have practical elements that are connected to the theoretical part. 
        Lucas did not enjoy the online courses he had followed last year. 
        Lucas stated that this was mostly due to motivational issues and unclear course structure. 
        Lucas thinks that we are currently focussing to much on simulating the offline learning 
        environment in an online setting. Lucas questions if this is the best way to get to ideal online education.
        \item[] \textbf{Jelle Witsen Elias: } 
        Jelle did not have any experiences with online learning so far. 
        Interstly, Jelle does not have a background in computer science but in law.
        This meant that the focus of the interview was more on his experiences with
        offline learning and how he thinks this will transfer over to an online platform.
        Jelle stated that motivation is an important part of a course for him and 
        that teachers should focus on this part. Jelle thinks that in an online 
        setting it is even more important for the teacher to interact with the student. 
    \end{itemize} 
    \item[] \textit{Reflection:} I think that the interviews turned out decent but 
    I do have some points I think I should improve on. The main thing is that I felt 
    that the three interviews resulted in very similar answers. This is because I 
    felt that the interviews stayed very surface level, and did not explore the 
    subject enough. Another point of improvement for me is ending the interview.
    I did not really know when to end the interviews. A possible way to fix these 
    problems is to prepare better questions and define my goal for the interview better. 
    \item[] \textbf{E.4}
    \item[] 
    \item[] \textbf{E.5}
    \item[] For the overview of Corona apps we have made a google Drive file located at
    \item[] \url{https://docs.google.com/document/d/1u6IUTjyClg6BgeTIaZAXrhK2G_hl9F87RYOgBVDjTw4/edit#heading=h.r3e7ka7zotug}
\end{itemize}

\end{document}