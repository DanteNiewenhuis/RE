\documentclass[]{article}
\usepackage[utf8]{inputenc}
\usepackage{hyperref}

\usepackage[
backend=biber,
style=numeric
]{biblatex}

\addbibresource{Portfolio.bib}

\title{Portfolio}
\author{Dante Niewenhuis}

\begin{document}

\maketitle

\section*{Week 1}

\subsection*{Reading} 

\subsubsection*{R.1} 
Following is a small summary of the text. Notes that I have taken during reading 
can be found on google Drive in the Notes folder.\\\\
\textbf{\textit{Summary:}} The mind operates using two systems. 
System 1 operates quick and automatically, while system 2 is used to answer 
more complex questions/activities and monitor system 1. System 2 is also in
charge of judging and unbelieving, but does this based on the information 
given by system 1. 
The problem can be however that system 2 is very lazy and lets system 1 do most of the work, only interfering 
when either the question is to difficult or the quick answer generated by 
system 1 turns out to be insufficient. System 1 operates using heuristics. 
Heuristics are methods of finding answers that are adequite but often imperfect.
One example of problems caused by heuristics are visual illusions.  

\subsubsection*{R.2}
Following is a small summary for each of the papers read. 
Notes that I have taken during reading can be found on google Drive in the Notes folder.

\begin{itemize}
    \item[] \textbf{\textit{Software Requirements and Specifications: }}
    Building software is like making a machine. The parts that will be interacted 
    with is the application domain. In general developers tend to combine the 
    requirements for the application and the machine. This is however often 
    problametic because in most cases the developers will only focus on the machine part.
    The author states that the requirement of the application domain should be 
    defined seperately and explorent extensively. The author also states that 
    the evolution of development methods has been harmed because the idea is 
    created that there is a single best method of development. 

    \item[] \textbf{\textit{Contextual Design: }}
    Contextual design is a design approach in which the designer is involved in all step.
    This also includes tasks like data collection. This helps the designer to 
    get to know what is important for the customer. The designer should start with 
    interviews, and focus on interviewing all types of people in the process, and not just 
    the experts. These interviews should be combined into consolidations. These
    consolidations should then be shared with all people that could provide with 
    good feedback. Another important thing is to redesign the workplace. 
    During this process it is decided what technology is essential for development.
    Another suggestion is the creation of An User Environment Design (UMD), which 
    shows each part of the system and all interactions with them. Lastly, the 
    author states that it is important to iterate a design early and often. A
    good way to do an initial test is to use paper prototyping.

    \item[] \textbf{\textit{Setting the Scene: }}
    To make sure that software solution correctly solves a particular problem, 
    we must first understand the problem. 
    The aim of a software project is to improve the world using a machine. 
    In a machine-building project, our business as requirements engineers is to 
    investigate the problem world. There are two versions of this world: System-as-is, 
    the system that exists before the machine and System-to-be, 
    the system when the machine is done. 

    A problem can be structured into three dimensions. 
    The first dimension is why. The why dimensions focusses on defining what the problem is 
    that the machine is going to alleviate. Common actions in the why dimension 
    are the following:
    \begin{itemize}
        \item[] Acquiring domain knowledge 
        \item[] Evaluating alternative options in the problem world 
        \item[] Evaluating technology opportunities 
        \item[] Handling conflicts
    \end{itemize}

    The second dimension is what. 
    This dimension is concerned with the functional services that the system-to-be 
    should provide. This part should be defined precisely and should be written in 
    such a way that all parties understand fully.  

    The third dimension is who. 
    This dimension defines who of the staff is responsible for specific tasks. 
    Decisions about responsibility are vital for a projects success. 


    \item[] \textbf{\textit{TaskModeling: }}
    Traditional functional requirements specify the system's role but ignore the context. 
    This is unsuitable because it prematurely divides work between the computer and the user.
    The author suggest making a work area description, which states formal requirements 
    and overal purpose. The author states that it is very important to determine 
    what are seperate tasks and what are subtasks. This will provide a much better 
    overview of the systems essentials and structure. 
    A good systematic way of expressing problems and solutions are 
    tasks and support descriptions. These disctriptions combine subtasks and 
    potential solutions. 
\end{itemize} 

\subsubsection*{R.3}
Sensemaking is the process through which people work to understand issues or events that are novel, 
ambiguous, confusing, or in some other way violate expectations \cite{maitlis2014sensemaking}.
Sensemaking is a critical organizational activity. For top managers, sensemaking 
activities are key tasks that significantly influence organizational decisions\cite{maitlis2005social}.
\citeauthor{maitlis2005social} states that methods of sensemaking can be devided into four
distinct forms: guided, fragmented, restricted and minimal. \cite{maitlis2005social}. 

The first type of sensemaking is Guided sensemaking. 
This type is highly controlled and highly animated. 
In guided sensemaking, the stakeholders had a high level of engagement due to 
many orginaized controlled discussions.
This results in higher knowledge of the issues.

The second type of sensemaking is Freagmented sensemaking. 
This type is animated but not controlled. This is similar to guided 
sensemaking with the stakeholders still being very involved, 
but without controlled discussions.

The thrid type is Restricted sensemaking, 
This type is very controlled but not animated. 
This is similar to guided sensmaking but instead of involving all stakeholders, 
only a few specific stakeholders are involved.

The last type of sensemaking is minimal sensemaking.
This type is neither animated or controlled. This meant that all parties 
in the organisation kept to themselves and only conclusions were shared between them.



\subsubsection*{R.4}
Online education is getting more and more important. In 2013 already 33.5\% of 
student were enrolled in at least one online course \cite{allen2014grade}. Another 
factor which makes online education so important is the global pandemic caused by 
the corona crisis. While online education is often seen as inferior to offline learning,
it has some advantages. Examples of advantages are the increased access of courses 
for students that live far away and the allivaiation of capacity constraints due to 
room sizes\cite{volery2000critical}. Online learning is however not only positive.
Examples of problems are teachers feeling less able to retain student attention,
students requirering more discipline to succeed and the loss of social interaction\cite{allen2014grade}.

What the ideal online course should look like is difficult to describe but I will 
try to define what I think are important factors for a good online course based on 
a small study, interviews with other students and my own experiences. 
From the interviews as well as my own experiences I conclude that structure is very 
important for online courses. Because online courses require the student to work 
more on their own it is important that the student should be able to get a clear 
idea of what is expected without much interaction. The second important factor in an 
online course is the tools that are used and how well a teacher uses it. A research 
into the success of online learning\cite{volery2000critical} states that lecturers should
adapt their lessons to make optimal use of the online tools available. Most importantly 
is focussing on the interaction with students. A research on the satisfaction of 
online enducation in China during the pandemic\cite{chen2020analysis} states that 
the biggest factor of satisfaction for students was the quality of the tools used.

As a conclusion, I think that the succes of an online course is dependent on how 
well the tools to give an organized and structured course. I think it is also important
that the lecturer focusses on the interactive parts of his lectures. 

\subsection*{Exercises}

\subsubsection*{E.1}
A small presentation on online learning can be found in my Drive

\subsubsection*{E.2}
My favourite interviewer is Louis Theroux. What I admire about Theroux is his 
ability to interview so many types of people. I think the reason he is so succesful
is that he makes his interviewees feel very confortable. One of the reasons I think 
he is able to do this is because seems very unjudging en passive. Even though he 
might not look or act vary daring, I think it is that kind of atmosphere that 
allows him to get away with asking questions that might be to extreme for other
interviewers to ask without any bad responses. These things make it possible for 
him to interview people that don't want to be interviewed by other people.

Another thing I like about Theroux is the type of questions he asked. Most 
of his questions are very simple, but his interviews still result in very 
interesting conversations. 

An example of Theroux getting away with asking saying something bad is in the following clip:
\url{https://www.youtube.com/watch?v=wUO2MBKICiE}.
Theroux is talking to the Phelps family, one of the most extreme religious people,
and at 3:30 suggest that the mother might not accualy believe what she is saying.

What I hope to emulate from Theroux is to also create a confortable environment
and ask simple but good questions. 

\subsubsection*{E.3}
The recordings of my interviews can be found on Drive in the interview folder. 
Following are what I found to be the most interesting thing from the interviews.
After this I will reflect on the interviews. 

\begin{itemize}
    \item[] \textbf{Clement Julia: }  
    Clement does not have any experiences with online learning so far. 
    This made it more difficult for us to talk specifically about online learning. 
    Instead we discussed what he finds important for courses in general. 
    Clement stated that structure in lectures is very important. 
    An example he gave to reach this is by having a clear first slide which shows 
    what will be discussed in the lecture and how this is connected to the rest of the course. 
    \item[] \textbf{Lucas Steehouwer: } 
    Lucas has experience with online while he was doing his previous master last year. 
    For Lucas it is important that courses have practical elements that are connected to the theoretical part. 
    Lucas did not enjoy the online courses he had followed last year. 
    Lucas stated that this was mostly due to motivational issues and unclear course structure. 
    Lucas thinks that we are currently focussing to much on simulating the offline learning 
    environment in an online setting. Lucas questions if this is the best way to get to ideal online education.
    \item[] \textbf{Jelle Witsen Elias: } 
    Jelle did not have any experiences with online learning so far. 
    Interstly, Jelle does not have a background in computer science but in law.
    This meant that the focus of the interview was more on his experiences with
    offline learning and how he thinks this will transfer over to an online platform.
    Jelle stated that motivation is an important part of a course for him and 
    that teachers should focus on this part. Jelle thinks that in an online 
    setting it is even more important for the teacher to interact with the student. 
\end{itemize} 
\textit{Reflection:} I think that the interviews turned out decent but 
I do have some points I think I should improve on. The main thing is that I felt 
that the three interviews resulted in very similar answers. This is because I 
felt that the interviews stayed very surface level, and did not explore the 
subject enough. Another point of improvement for me is ending the interview.
I did not really know when to end the interviews. A possible way to fix these 
problems is to prepare better questions and define my goal for the interview better. 

\subsubsection*{E.4}
I have reviewed the interviews of Ivan Veno. In my review I will mainly focus 
on how the interaction with the interviewee is, what kind of questions you ask 
and if you can get all the information out of the interview that you might want. 
I will start with my overal impressions and follow it up with some snippets which 
I found interesting. The interviews can be found using the following links:
\begin{itemize}
    \item[] \textbf{David: } \url{https://www.youtube.com/watch?v=fLVJzaw_s0U&feature=emb_title}
    \item[] \textbf{Wilco: } \url{https://www.youtube.com/watch?v=W6CUbnptKhY&start=135s}
    \item[] \textbf{Joachim: } \url{https://www.youtube.com/watch?v=7HV3gSq9D1c}
\end{itemize}
My general impression of Ivan's interviews are very positive. The interviews
flow well and everyone seems to be very confortable. I like how Ivan managed 
to get much information out of the interviewees with few questions. 
I also liked how Ivan was reacting to the answers and turning them into new questions
The only problem I have with the interviews is that they felt somewhat short, especially the interview with David. 
Another small problem I have is that Ivan sometimes summarizes the answers quite extensively which can effect the flow of the interview.
All in all I enjoyed the interviews.\\
\textbf{\textit{snippets}}
\begin{itemize}
    \item [] \textit{David: 2:54} good continuation of his answer into a new question
    \item [] \textit{Wilco: 2:35} good continuation of his answer into a new question
    \item [] \textit{Joachim: 2:16} Joachim just tanked about his hard times with being motivated 
                in the first few lessons. A good follow up question here might be if he has any suggestions
                on how to fix this problem
\end{itemize} 

\subsubsection*{E.5}
For the overview of Corona apps we have made a google Drive file located at
\url{https://docs.google.com/document/d/1u6IUTjyClg6BgeTIaZAXrhK2G_hl9F87RYOgBVDjTw4/edit#heading=h.r3e7ka7zotug}


\printbibliography
\end{document}

