\documentclass[]{article}
\usepackage[utf8]{inputenc}
\usepackage{hyperref}
\usepackage[shortlabels]{enumitem}

\usepackage[
backend=biber,
style=numeric
]{biblatex}

\addbibresource{Portfolio.bib}

\title{Portfolio}
\author{Dante Niewenhuis}

\begin{document}
\maketitle
\tableofcontents

\noindent\textbf{Drive: } \url{https://drive.google.com/drive/folders/1exf09LYeRfVdYIzBnLlPVWMENsfTiStx?usp=sharing}

\addcontentsline{toc}{section}{Week 1}
\section*{Week 1}

\addcontentsline{toc}{subsection}{Reading}
\subsection*{Reading} 

\subsubsection*{R.1} 
Following is a small summary of the text. Notes that I have taken during reading 
can be found on google Drive in the Notes folder.\\\\
\textbf{\textit{Summary:}} The mind operates using two systems. 
System 1 operates quick and automatically, while system 2 is used to answer 
more complex questions/activities and monitor system 1. System 2 is also in
charge of judging and unbelieving, but does this based on the information 
given by system 1. 
The problem can be however that system 2 is very lazy and lets system 1 do most of the work, only interfering 
when either the question is to difficult or the quick answer generated by 
system 1 turns out to be insufficient. System 1 operates using heuristics. 
Heuristics are methods of finding answers that are adequate but often imperfect.
One example of problems caused by heuristics are visual illusions.  

\subsubsection*{R.2}
Following is a small summary for each of the papers read. 
Notes that I have taken during reading can be found on google Drive in the Notes folder.

\begin{itemize}
    \item[] \textbf{\textit{Software Requirements and Specifications: }}
    Building software is like making a machine. The parts that will be interacted 
    with is the application domain. In general developers tend to combine the 
    requirements for the application and the machine. This is however often 
    problematic because in most cases the developers will only focus on the machine part.
    The author states that the requirement of the application domain should be 
    defined separately and explored extensively. The author also states that 
    the evolution of development methods has been harmed because the idea is 
    created that there is a single best method of development. 

    \item[] \textbf{\textit{Contextual Design: }}
    Contextual design is a design approach in which the designer is involved in all step.
    This also includes tasks like data collection. This helps the designer to 
    get to know what is important for the customer. The designer should start with 
    interviews, and focus on interviewing all types of people in the process, and not just 
    the experts. These interviews should be combined into consolidations. These
    consolidations should then be shared with all people that could provide with 
    good feedback. Another important thing is to redesign the workplace. 
    During this process it is decided what technology is essential for development.
    Another suggestion is the creation of An User Environment Design (UMD), which 
    shows each part of the system and all interactions with them. Lastly, the 
    author states that it is important to iterate a design early and often. A
    good way to do an initial test is to use paper prototyping.

    \item[] \textbf{\textit{Setting the Scene: }}
    To make sure that software solution correctly solves a particular problem, 
    we must first understand the problem. 
    The aim of a software project is to improve the world using a machine. 
    In a machine-building project, our business as requirements engineers is to 
    investigate the problem world. There are two versions of this world: System-as-is, 
    the system that exists before the machine and System-to-be, 
    the system when the machine is done. 

    A problem can be structured into three dimensions. 
    The first dimension is why. The why dimensions focusses on defining what the problem is 
    that the machine is going to alleviate. Common actions in the why dimension 
    are the following:
    \begin{itemize}
        \item[] Acquiring domain knowledge 
        \item[] Evaluating alternative options in the problem world 
        \item[] Evaluating technology opportunities 
        \item[] Handling conflicts
    \end{itemize}

    The second dimension is what. 
    This dimension is concerned with the functional services that the system-to-be 
    should provide. This part should be defined precisely and should be written in 
    such a way that all parties understand fully.  

    The third dimension is who. 
    This dimension defines who of the staff is responsible for specific tasks. 
    Decisions about responsibility are vital for a projects success. 


    \item[] \textbf{\textit{TaskModeling: }}
    Traditional functional requirements specify the system's role but ignore the context. 
    This is unsuitable because it prematurely divides work between the computer and the user.
    The author suggest making a work area description, which states formal requirements 
    and overall purpose. The author states that it is very important to determine 
    what are separate tasks and what are subtasks. This will provide a much better 
    overview of the systems essentials and structure. 
    A good systematic way of expressing problems and solutions are 
    tasks and support descriptions. These descriptions combine subtasks and 
    potential solutions. 
\end{itemize} 

\subsubsection*{R.3}
Sensemaking is the process through which people work to understand issues or events that are novel, 
ambiguous, confusing, or in some other way violate expectations \cite{maitlis2014sensemaking}.
Sensemaking is a critical organizational activity. For top managers, sensemaking 
activities are key tasks that significantly influence organizational decisions\cite{maitlis2005social}.
\citeauthor{maitlis2005social} states that methods of sensemaking can be divided into four
distinct forms: guided, fragmented, restricted and minimal. \cite{maitlis2005social}. 

The first type of sensemaking is Guided sensemaking. 
This type is highly controlled and highly animated. 
In guided sensemaking, the stakeholders had a high level of engagement due to 
many organized controlled discussions.
This results in higher knowledge of the issues.

The second type of sensemaking is Fragmented sensemaking. 
This type is animated but not controlled. This is similar to guided 
sensemaking with the stakeholders still being very involved, 
but without controlled discussions.

The third type is Restricted sensemaking, 
This type is very controlled but not animated. 
This is similar to guided sensmaking but instead of involving all stakeholders, 
only a few specific stakeholders are involved.

The last type of sensemaking is minimal sensemaking.
This type is neither animated or controlled. This meant that all parties 
in the organization kept to themselves and only conclusions were shared between them.


\subsubsection*{R.4}
Online education is getting more and more important. In 2013 already 33.5\% of 
student were enrolled in at least one online course \cite{allen2014grade}. Another 
factor which makes online education so important is the global pandemic caused by 
the corona crisis. While online education is often seen as inferior to offline learning,
it has some advantages. Examples of advantages are the increased access of courses 
for students that live far away and the alleviation of capacity constraints due to 
room sizes\cite{volery2000critical}. Online learning is however not only positive.
Examples of problems are teachers feeling less able to retain student attention,
students requiring more discipline to succeed and the loss of social interaction\cite{allen2014grade}.

What the ideal online course should look like is difficult to describe but I will 
try to define what I think are important factors for a good online course based on 
a small study, interviews with other students and my own experiences. 
From the interviews as well as my own experiences I conclude that structure is very 
important for online courses. Because online courses require the student to work 
more on their own it is important that the student should be able to get a clear 
idea of what is expected without much interaction. The second important factor in an 
online course is the tools that are used and how well a teacher uses it. A research 
into the success of online learning\cite{volery2000critical} states that lecturers should
adapt their lessons to make optimal use of the online tools available. Most importantly 
is focussing on the interaction with students. A research on the satisfaction of 
online education in China during the pandemic\cite{chen2020analysis} states that 
the biggest factor of satisfaction for students was the quality of the tools used.

As a conclusion, I think that the success of an online course is dependent on how 
well the tools to give an organized and structured course. I think it is also important
that the lecturer focusses on the interactive parts of his lectures. 

\addcontentsline{toc}{subsection}{Exercises}
\subsection*{Exercises}

\subsubsection*{E.1}
A small presentation on online learning can be found in my Drive

\subsubsection*{E.2}
My favorite interviewer is Louis Theroux. What I admire about Theroux is his 
ability to interview so many types of people. I think the reason he is so successful
is that he makes his interviewees feel very comfortable. One of the reasons I think 
he is able to do this is because seems very unjudging en passive. Even though he 
might not look or act vary daring, I think it is that kind of atmosphere that 
allows him to get away with asking questions that might be to extreme for other
interviewers to ask without any bad responses. These things make it possible for 
him to interview people that don't want to be interviewed by other people.

Another thing I like about Theroux is the type of questions he asked. Most 
of his questions are very simple, but his interviews still result in very 
interesting conversations. 

An example of Theroux getting away with asking saying something bad is in the following clip:
\url{https://www.youtube.com/watch?v=wUO2MBKICiE}.
Theroux is talking to the Phelps family, one of the most extreme religious people,
and at 3:30 suggest that the mother might not actually believe what she is saying.

What I hope to emulate from Theroux is to also create a comfortable environment
and ask simple but good questions. 

\subsubsection*{E.3}
The recordings of my interviews can be found on Drive in the interview folder. 
Following are what I found to be the most interesting thing from the interviews.
After this I will reflect on the interviews. 

\begin{itemize}
    \item[] \textbf{Clement Julia: }  
    Clement does not have any experiences with online learning so far. 
    This made it more difficult for us to talk specifically about online learning. 
    Instead we discussed what he finds important for courses in general. 
    Clement stated that structure in lectures is very important. 
    An example he gave to reach this is by having a clear first slide which shows 
    what will be discussed in the lecture and how this is connected to the rest of the course. 
    \item[] \textbf{Lucas Steehouwer: } 
    Lucas has experience with online while he was doing his previous master last year. 
    For Lucas it is important that courses have practical elements that are connected to the theoretical part. 
    Lucas did not enjoy the online courses he had followed last year. 
    Lucas stated that this was mostly due to motivational issues and unclear course structure. 
    Lucas thinks that we are currently focussing to much on simulating the offline learning 
    environment in an online setting. Lucas questions if this is the best way to get to ideal online education.
    \item[] \textbf{Jelle Witsen Elias: } 
    Jelle did not have any experiences with online learning so far. 
    Interestedly, Jelle does not have a background in computer science but in law.
    This meant that the focus of the interview was more on his experiences with
    offline learning and how he thinks this will transfer over to an online platform.
    Jelle stated that motivation is an important part of a course for him and 
    that teachers should focus on this part. Jelle thinks that in an online 
    setting it is even more important for the teacher to interact with the student. 
\end{itemize} 
\textit{Reflection:} I think that the interviews turned out decent but 
I do have some points I think I should improve on. The main thing is that I felt 
that the three interviews resulted in very similar answers. This is because I 
felt that the interviews stayed very surface level, and did not explore the 
subject enough. Another point of improvement for me is ending the interview.
I did not really know when to end the interviews. A possible way to fix these 
problems is to prepare better questions and define my goal for the interview better. 

\subsubsection*{E.4}
I have reviewed the interviews of Ivan Veno. In my review I will mainly focus 
on how the interaction with the interviewee is, what kind of questions you ask 
and if you can get all the information out of the interview that you might want. 
I will start with my overall impressions and follow it up with some snippets which 
I found interesting. The interviews can be found using the following links:
\begin{itemize}
    \item[] \textbf{David: } \url{https://www.youtube.com/watch?v=fLVJzaw_s0U&feature=emb_title}
    \item[] \textbf{Wilco: } \url{https://www.youtube.com/watch?v=W6CUbnptKhY&start=135s}
    \item[] \textbf{Joachim: } \url{https://www.youtube.com/watch?v=7HV3gSq9D1c}
\end{itemize}
My general impression of Ivan's interviews are very positive. The interviews
flow well and everyone seems to be very comfortable. I like how Ivan managed 
to get much information out of the interviewees with few questions. 
I also liked how Ivan was reacting to the answers and turning them into new questions
The only problem I have with the interviews is that they felt somewhat short, especially the interview with David. 
Another small problem I have is that Ivan sometimes summarizes the answers quite extensively which can effect the flow of the interview.
All in all I enjoyed the interviews.\\
\textbf{\textit{snippets}}
\begin{itemize}
    \item [] \textit{David: 2:54} good continuation of his answer into a new question
    \item [] \textit{Wilco: 2:35} good continuation of his answer into a new question
    \item [] \textit{Joachim: 2:16} Joachim just tanked about his hard times with being motivated 
                in the first few lessons. A good follow up question here might be if he has any suggestions
                on how to fix this problem
\end{itemize} 

\subsubsection*{E.5}
For the overview of Corona apps we have made a google Drive file located at
\url{https://docs.google.com/document/d/1u6IUTjyClg6BgeTIaZAXrhK2G_hl9F87RYOgBVDjTw4/edit#heading=h.r3e7ka7zotug}

\clearpage
\addcontentsline{toc}{section}{Week 2}
\section*{Week 2}

\addcontentsline{toc}{subsection}{Reading}
\subsection*{Reading} 

\subsubsection*{R.1}
Following is a small summary of the text. Notes that I have taken during reading 
can be found on google Drive in the Notes folder.\\

\textbf{\textit{Summary: }} 
People are not great in handling statistics. System 1 cannot properly understand 
statistical facts, which leads to many bad conclusions. Extreme outcomes are 
more common when dealing with smaller samples. People often tend to create patterns
in random events, because system 1 prefers certainty over doubt. This creates 
what is called the law of small numbers. Small probabilities are handled by 
people in two ways: either they negate them, or they value it way to strongly.

People use other numbers to anchor to, even when these numbers are not related. 
anchoring means that the answer to a question is similar to another number, this 
number is the anchor. While anchoring is something done by system 2, it still 
inherits the biases of system 1. 

People have biases when judging situations. For instance stating what is a higher
cause of death is very difficult. Another problem with the human mind is the 
that we have difficulty understanding probability. An example of this is Linda,
a woman who can either be a "bank teller" or a "feminist bank teller". The second 
is a subset of the first, so the first is always the correct answer. This is 
however not how all people answer.

When dealing with probability people also tend to give much more weight to specific 
information about a case compared to the base rate. A good way to not fall into 
this is to first set a base rate by making an average assumption and then 
deviate from this assumption using the given information.


\subsubsection*{R.2}
Following is a small summary for each of the papers read. 
Notes that I have taken during reading can be found on google Drive in the Notes folder.

\textbf{\textit{Apprenticing with the Customer: }}
In this paper the author discusses the use of a master apprentice relationship 
for the designer. This type of relationship works well because it does not require
the costumer to be an expert at explaining himself. This method should lead the 
designer to have a better understanding of the situation he is building software for.
   

\subsubsection*{R.4}
\textbf{\textit{What Great Listeners Actually Do: }}
People often think that they are good listeners, but this is mostly false.
To be a good listener one should ask goo questions, include interactions that
build self-esteem, creat a cooperative conversation and make good suggestions.

\subsubsection*{R.6}
Being a good listener is more than just being quiet and occasionally ask a question.
\citeauthor[]{zenger2016great} state that a good listening is a process in which 
the listener engages in the conversation by periodically asking questions, 
making interactions that build the speakers self-esteem and making
good suggestions \cite{zenger2016great}. Methods of becoming a better listener are 
clearing away distractions like phones or trying to observe nonverbal cues, 
such as facial expressions. \citeauthor[]{grohol2018become} adds that methods 
to show that you are actively listening include summarizing, paraphrasing or 
reflecting on what the speaker just said \cite{grohol2018become}. Aside from 
only being a good a good listener it could also show the speaker what parts 
of the story where not clear enough. This could cause the speaker to elaborate 
on what he just said. In this paper \citeauthor[]{grohol2018become} also states 
that it is better to ask for elaboration in a friendly matter, if your reply 
is questioning their answer to much, it can make the speaker defensive. 
\citeauthor[]{grohol2018become} also states that asking for an example when
an example can make the speaker elaborate on their point. This is similar to 
tactics described by \citeauthor[]{beyer1995apprenticing}, who states that 
a designer should ask simple questions and focus on examples because you cannot
expect a costumer or interviewee to be able to answer questions in abstract form \cite{beyer1995apprenticing}.

\subsubsection*{R.7}
As stated in \textit{R.6}, there are many ways of being a good listener and 
ensuring an environment in which the speaker feels comfortable and easy to 
talk in depth about the questions at hand. There are however also some thing that 
are best to be avoided if one wants to be a good listener. \citeauthor[]{beyer1995apprenticing}
states in his paper that there are seven roadblocks which should be avoided \cite{beyer1995apprenticing}.
Those roadblocks are the following:
\begin{enumerate}
    \item asking to many "Why" questions. This can make the speaker defensive
    \item Quick reassurances, like "Don't worry about that"
    \item Advising, while sometimes appropriate, can also make for bad conversations
    \item Digging for information when the speaker would rather not give it
    \item Being patronizing
    \item Being preaching
    \item Interrupting the speaker
\end{enumerate}

These things can all break the flow of the conversation and create an environment
in which the speaker would rather be quiet.

\subsubsection*{R.8}
Memorizing is an important human ability. I will discuss different techniques
of memorization for the task of learning word vocabulary, but I think these 
methods could be adapted for much more tasks.
In their paper, \citeauthor{oxford1990vocabulary} 
look at the influence of memorization techniques for learning vocabulary \cite{oxford1990vocabulary}. 
The memorization techniques are classified into four groups: de-contextualizing, 
semi-contextualizing, fully contextualizing and adaptable.

The first group of methods is de-contextualizing. This means that the words 
are removed from context fully. Examples of this technique are word-lists or 
flashcards. This method is very common in high school education, but has several 
issues. The first issue is that students learn list of words for when they are 
tested, and forgets them immediately after. The other problem is that because of 
the lack of context students often create their on associations between word 
that are not valid at all. 

The second group of methods is semi-contextualizing. Semi-contextualizing techniques allow
some degree of context but fall short of full context. Examples of this technique
are \textit{word groups}, which means dividing lists of groups into groups of similar meaning,
and \textit{word or Concept Association}, which means that new words are 
connected to already known word to provide context.

The third group of methods is fully contextualizing. Fully contextualizing focuses on 
embedding the new words fully in their normal context. Examples of this technique
are reading and Listening practice, using bigger stories, or speaking and writing practice. 
This method has the advantage that students will retain vocabulary just through 
reading practice and will made them more aware of context. The problem with this 
method however is that while students might be able to determine the meaning of 
a word in context, there is no guarantee that the word is completely learned or known.

The fourth group of methods is adaptive. The adaptive method reinforces the other 
techniques at the part of the contextuality continuum. The adaptive method entails 
going back to words that were already learned at different intervals. This 
constant repetition is key to memorizing word. 

Memorizing is best done by combining different styles of techniques, both 
contextually and non-contextually. Frequent repetition of the task at hand is 
also important to create a good memory.

\addcontentsline{toc}{subsection}{Exercises}
\subsection*{Exercises} 

\subsubsection*{E.1}
This is an exercise that will be done together with all other students. 
My task for this week to get an overview of our group of students. 
We have decided to do this using a big poll which hopefully highlights the 
characteristics of the masters Software Engineering student. 

The results are evaluated and can be found on the following drive:
\url{https://drive.google.com/drive/folders/1LU9RBTQrom6dJG54GUKyKW5r3XZvCH9o?usp=sharing}.

\subsubsection*{E.2}
I think the problem with the goal of every student evaluating the course with a 
10 is that every students definition of a 10 is different. For some students it 
means the course being fun, for others it means learning a lot, and for 
others it will mean a course without problems. The question is if this means that 
the grade of a course is thereby the best measure of success.

\subsubsection*{E.3}
For this weeks interviews I will be talking with Lucas Steehouwer and Clement Julia.
We will be talking about what types of jobs we want to get after our masters.
Last week I felt that I did not got deep enough into the question as
I would have liked. This is why this week I have made a more detailed plan of action.
The general plan for the interviews is the following:\\
First I will ask the interviewee about past jobs, what they liked, 
and what they didn't like.
If they have done jobs related to our field of study I will ask them more elaborate 
questions about it. 
If they haven't done any jobs yet, or very little, I will skip this part and 
try to focus more on other parts.

In the second part I will ask if the interviewee already has an idea of a future
job. If so, I will ask him to elaborate, and ask him how he knows this so sure. 

The next part is talking about what the interviewee finds important for a job 
and why. I will also ask if the interviewee already has a goal for a job in the 
further future. I will also ask if he thinks the job he will take just after the 
master is different from what he will do later down the line. 

I will end my interview with giving the interviewee a set of choices. Each 
will be a choice between two things for which I would like to know which one they
prefer, how much they prefer it, and why.\\

The goal of the interviews is to get a good idea of the aspirations of the 
interviewees. \citeauthor[]{castillo2016preparing} states in her research that 
during research interviews we should not ask pure research questions because they
would overwhelm the interviewee \cite{castillo2016preparing}. 
This is why will try to ask as little abstract questions, 
and end the interview with choice questions which will make the ideas more palpable. 
The step of making the process less abstract is what I got from the paper 
explained in \textit{R.2}. The questions can be found on the drive.

\begin{itemize}
    \item[] \textbf{Clement Julia: }
    The first interview was done with Clement. Clement has worked for a big company
    full time as part of his bachelors. He enjoyed it and found that he had learned a lot 
    from it. He did state that he did not like the amount of things he had to do 
    that was not related to programming.

    For the near future Clements states that he first would like to go to a smaller
    company and maby later transition back to a bigger company. Clement does not
    have solid plan for after the masters, just that he would like to work at 
    a smaller company. It is important for Clement to have freedom in what to do,
    something he felt was lacking in his last job. 

    For the further future he expects to eventually go over to a larger company.
    Clement enjoys programming so he would like to be more technical than social 
    in his work. 

    At the end Clement states that he would like to work at a startup rather 
    than an established company because it will provide more freedoms. 
    Finally Clement states that motivation is more important than money, 
    given that the salery will still be enough to live on. 
    
    \item[] \textbf{Lucas Steehouwer: }  
    My second interview was done with Lucas. Lucas did not have much work experience
    in the field of our study aside from being a TA during his bachelors. 
    Lucas liked being a TA but not for the courses with bad organization.  

    Lucas has now good plans for a job after his masters for a job. 
    He does not have things he really wants to do, but has some examples of things 
    he would not like to do. Lucas explained that he started with the bachelor 
    information science and later switched to computer science. He said he 
    made the change because he did not want to become a consultant, which 
    many information science students become. Lucas also states that he does not 
    want to get involved with the financial world, because he thinks it is 
    uninteresting and unethical. 

    Lucas states that he does not mind being at the same company for a long time
    because he likes the stability. Lucas also states that he would not want to 
    work in a company that is too large since he does not like the organizational 
    work it brings with it.

\end{itemize}

\noindent\textbf{Reflection: }
I am satisfied with the interviews of this week. I think the interviews were 
better than last week. I think that the improvement was caused by a better 
plan beforehand. I do think that the plan could be improved even more. 
One of the things that I found was that some questions seemed to have the same 
purpose. This meant that the question was made redundant by the answer to a 
previous question. I think that focusing on what each question will add 
to the interview beforehand could improve the overall interview.

\subsubsection*{E.4}
For this weeks review I have looked at Lucas Steehouwer interviewing Clement Julia.
I will review the interview looking at the following elements:
\begin{itemize}
    \item Are the base questions answered
    \item How deep is the interview diving in the subject
    \item How is the interaction with the interviewee
\end{itemize}

\noindent I would like to start with stating that I really enjoyed the interview.
I think the interaction between the interviewer and interviewee seemed good and 
the interview had a good flow. One thing I was somewhat missing was a question 
if the interviewee had a certain aspect of coding in which he would really like 
to work. In general I think the interview worked very well. Next, I will list 
some timestamps which I found particularly interesting:

\begin{itemize}
    \item[] \textit{8:20} I really like this question because it gives us some 
    insight and is a good reactions to what Clement has said during the interview. 
    \item[] \textit{7:10} Clement start talking about him wanting to still be 
    relevant in 10 years time. A good question in my opinion would be to ask 
    if he has a plan to make sure this is the case. 
\end{itemize}




\subsubsection*{Ad Active listening}

\subsubsection*{Ad Reliability of information you get ou of interviews}

\addcontentsline{toc}{section}{Week 3}
\section*{Week 3}

\addcontentsline{toc}{subsection}{Reading}
\subsection*{Reading} 

\subsubsection*{R.1} 
Following is a small summary of the text. Notes that I have taken during reading 
can be found on google Drive in the Notes folder.\\\\
\textbf{\textit{Summary:}} 
Humans constantly convince themselves to have a good understanding of future and 
past. Humans have a hindsight bias. This means that they evaluate something based 
on the outcome and not on the soundness of the process. This is problematic 
because it rewards irresponsible risk seekers that are successful by luck. 

System 1 makes us thing see the world as more tidy and simple than it actually is.
This gives us the illusion that we understand the past. The problem is that this 
makes it hard for us to distinguish the influence of sound decisions and luck.

Another interesting thing stated by the author was that humans are often overconfident.
This was shown in multiple ways but mainly in the form of stock market traders. 
Most traders aren't performing consistent or well, but are still very 
confident in their skills.

\subsubsection*{R.3}
People can react to knowing very little about a subject in different ways. Some 
people will try to first get more knowledgeable about the subject, or will 
not comment at all because they know they have not enough knowledge on the 
subject. However, there is also another common reaction of people called
Ultracrepidarianism. Ultracrepidarianism is a concept that means giving opinions 
or advise about things that one knows very little about. 

Ultracrepidarianism can be dangerous when the comments made can have bad consequences.
An example of this is the current Covid 19 crisis. \citeauthor[]{Villain2020} 
states in his article that the current crisis has shown many people commenting 
on this complex issue without any real knowledge \cite{Villain2020}. He thinks 
that there are two reasons for this phenomenon. The first reason is the media
overflowing the people with information. The media's goal is almost always to 
have more viewers, which means that not all information will be shown equally.
The second reason is that there is still scientific uncertainty. This uncertainty
causes people to think that every theory is just as valid as the other. The 
Covid crisis is a great example of the dangers of Ultracrepidarianism.

Another concept which is related to Ultracrepidarianism is the Dunning-Kruger
effect. The Dunning-Kruger can be described as the ignorance of ones 
ignorance \cite{DUNNING2011247}. \citeauthor{DUNNING2011247} show in their paper
that the less knowledgeable someone is of subject the more likely they are to 
misjudge how knowledgeable they are. Their deficits leave them with a double 
burden—not only does their incomplete and misguided knowledge lead them to make 
mistakes but those exact same deficits also prevent them from recognizing when 
they are making mistakes and other people choosing more wisely.

Another somewhat similar effect was discussed in part 3 of \textit{thinking, fast and slow} 
\cite{kahneman2011thinking}. In this part \citeauthor{kahneman2011thinking} 
discusses people being very confident in their decisions even after being 
shown to be not very good. He shows that even he was suspectable to this. 

It seems that while some people can deal with not knowing some things and letting
other people make decisions, many people convince themselves that they know enough
to answer themselves.

\subsubsection*{R.4}
Humans are often prone to remembering things different than how they actually 
happened. Many forms of memories can be untrue and can be altered. One example 
of short term memory being influenced was shown by \citeauthor{lampinen1997memory}
\cite{lampinen1997memory}.
They did this by first presenting the participants a list of words. After, the first list 
was removed and a second list of words was presented. The participants were 
asked to identify which words from the second list were present in the first list.
The authors found that when the second list consisted of words that had similar 
meaning, the participants would perform much worse. This means that the 
memory was altered to fit words in the list that were not in them before.

Another way of creating false memories is during traumatic experience. In her 
Ted talk \citeauthor{Ted:1} states that memories are stored better in different
levels of stress. Memories are stored the best if under a reasonable amount of stress.
However, when the stress levels get higher than a reasonable level, the creation
of memories gets more difficult. This means that people with traumatic experiences
often either don't remember it, or incorrect.

Another way of false memories is called confabulation. \citeauthor{confabulation}
define confabulation as honest lying. Confabulation is common with Alzheimer 
patients. The patient that is suffering from confabulation is unaware of the 
falsehoods, and will cling to their false beliefs. 

An interesting other concept of false memory is called the Mandela effect. 
The Mandela effect is a false memory shared by a group of people. The 
example were this effect gets is name from is that there is a general idea 
that Nelson Mandela has died in the 80s. This is however false since Nelson Mandela
died in 2013. In his paper \citeauthor{french2019mandela} states that the Mandela
effect is heavily linked to conspiracy theories.

The last thing I wanted to discuss was Part 3 of \textit{thinking, fast and slow}
\cite{kahneman2011thinking}. In this part the author states that it is harder for 
us to understand why we made a decision, after we know the outcome of that 
decision. This is not directly false memory but does mean that we cannot always 
understand our past self completely. 


\addcontentsline{toc}{subsection}{Exercises}
\subsection*{Exercises} 

\subsubsection*{E.1}
\begin{enumerate}[\textbf{a.}]
    \item I think the information from all different sections are very interesting 
    and broad. I have very little complains. The only thing I would have liked 
    to see is more conclusions based on the information gathered. An example of 
    this is Team 3, who were looking at the best online courses. While they have 
    discussed several courses in depth, I would have liked to see a final section
    in which the findings from the different online courses were combined.
    \item I think a key question for online learning should be "How can you 
    motivate a student during an online session". Another question I am very 
    interested in is "How can you organize you course in such a way that all 
    students can easily follow it from home."
    \item I think the best way of developing meaningful questions is by first 
    doing research on similar projects/problems. This will hopefully provide 
    the context with which you can more easily pinpoint problematic sections. 
    These problematic sections can than be examined further to create a set 
    of questions that will help solve the problems. 
    \item To answer my two questions I think it is first important to define general
    methods which are used to combat the problems at hand. The second step is to 
    think about how these methods can be applied in an online setting. 
    This will hopefully result in a list of possible solutions.
\end{enumerate}

\subsubsection*{E.2} \label{3.E.2}
I am part of the Info Support group. This means that we have the following goal:
"Help Info Support improve their website".\\
The Info Support group consist of Gert, Paul, Paulo, Tarick and myself. 
Our first step was to get an interview with Barbara, our contact from Info Support.
The interview was conducted by Gert and myself and following is a summary of 
this interview can be found at \url{https://docs.google.com/document/d/1uW1mdOzOwStVafZSbXqCa7DbdcaIacoP8U3i4agEKkg/edit}\\

\noindent The main lessons from this interview are the following:
\begin{itemize}
    \item The first interesting thing is that the employees are more like consultants,
    than just developers. This is something that is not clear from their website,
    and thus a possibility for improvement.
    \item The recruitment is currently a complex system which seems to work well
    since they have a very low number of students drop out
    \item Barbara stated that the website is updated frequently.
    \item Barbara also stated that while they get many students, it is mostly 
    through other website which link through.
\end{itemize}

\noindent Possible questions for the next interview:
\begin{itemize}
    \item How is the website designed to get people that want to have a job which 
    is like a consultant?
    \item Is the goal for Info Support to get more students that apply, or that 
    less students apply, but that they are a better match. This would cause less 
    rejections. 
\end{itemize}

\addcontentsline{toc}{section}{Week 3}
\section*{Week 4}

\addcontentsline{toc}{subsection}{Reading}
\subsection*{Reading} 

\subsubsection*{R.1} 
Following is a small summary of the text. Notes that I have taken during reading 
can be found on google Drive in the Notes folder.\\\\
\textbf{\textit{Summary:}} 
The problem with cognitive illusions is that they are much more stubborn than 
visual illusion. People are able to learn from past events but do not easily 
change their believes. An example the author gives is a research in which 
14 clinicians were compared to a simple algorithm. In 11 out of 14 cases the 
algorithm performed better. Even though this was substantial evidence that 
algorithms work better for some task none of the clinicians were convinced. 
The author states that humans often prefer humans over computer, even when 
the computer performs much better.  

The author continues with a story in which he shows that in many occasions just 
following a written out process for a task is the best you can do. The example 
he uses is of hiring people. The best thing you can do is to beforehand determine 
a few traits which are important and only focus on those for every interview. 
This is not a difficult process but just requires discipline.

The author asks himself when you can trust an expert intuition. The answer is 
that it depends on how regular the situation is, and how many opportunities the 
expert has had to learn about the situation.

The last part of this week discusses the difference between the inside and the 
outside view when working on a project. The inside view looks at the own 
experiences of the worker and forecasts the results of the projects based on 
them. The outside view means that you look at similar projects and what results 
they achieved. Only looking at the inside view often results in too optimistic
expectations.

\subsubsection*{R.2} 
\begin{enumerate}[a]
    \item TODO
    \item TODO
    \item TODO
\end{enumerate}


\addcontentsline{toc}{subsection}{Exercises}
\subsection*{Exercises} 

\subsubsection*{E.0}
With everybody taking the course we are collecting questions for Sven van der Zee.
The questions can be found at: \url{https://docs.google.com/document/d/12Cg9o90rfm9ZM15XyK_gVCbn3rpGS0q-3sg4W8x-czY/edit?usp=sharing}


\subsubsection*{E.1} 
\begin{enumerate}[a]
    \item I am part of the Info Support Group. We have the assignment from Info 
    Support to look at their website and think about how it could be improved 
    using requirements engineering techniques. This website is a possible career
    tool, which means that examining will be good for this exercise. 

    Last week we started with this assignment and so far have done two interviews.
    Information on these interviews can be found in \autoref{3.E.2}.  
    \item Because I am part of the Info Support group the plan is for us all to 
    interview a worker from Info Support. In this interview the plan is to get an 
    idea of the working background of the interviewee and how the other jobs 
    compare to his current job at Info Support. We will also ask the interviewee 
    for their opinions on the website. Important is for us to know if the core 
    principles of Info Support are clear from the website. An overview of the 
    plan and the questions can be found at: \url{https://docs.google.com/document/d/1MI_Xpl6SapmVF4h9q3UAOB1i28v9YZzRZZ_CwyKINKk/edit?usp=sharing}

    Sadly, we were not able to get enough people from Info Support to interview. 
    This means that I will be interviewing someone outside of Info Support. The 
    plan for the interview will be similar but I will skipping the questions 
    about Info Support. The overview of my plan for the interview can be found 
    at: \url{https://docs.google.com/document/d/1gZajR84-lH9N9xZllb95QI40PBV69NodAIiXWg3C2x8/edit?usp=sharing}
\end{enumerate}

\subsubsection*{E.2}
I will now describe two Requirements Engineering scenarios. I will describe what
the main setup is and what main problem is. I will then discuss the possible 
solutions and how requirements engineering could be used to get the best outcome.
\begin{enumerate}
    \item My first scenario is in the field of AI. What is interesting about AI 
    is that most companies have data but don't really know how to use it. All 
    companies want to do something with AI but only because it is the hip new 
    thing. In my example we are asked to help an auto car lease company. The 
    company only leases electric cars, which means that the cars gather data. 
    They want you use this data to better determine what type of contract fits 
    the driver. The challenge of this scenario is that the cars are gathering 
    a lot of data and determining what would be best to use is tricky. This is 
    also an example of a scenario were a complex result might not necessarily 
    be the best one.
    Besides this it is also important that the workers in the company understand 
    what you have done with the data and that it fits in the the working 
    environment. 
    \item 
\end{enumerate}

\subsubsection*{E.3}
In the Netherlands a corona app is developed. This app is created to combat the 
spread of Covid 19. I will now discuss two requirements of the dutch corona 
app and why I think they will fail. Note that I am only looking at the dutch
app and will assume that the app will not be mandatory. 

\begin{enumerate}
    \item \textit{People will be able to warn all other people they came close to
    if they get corona: } The idea of a corona app is that people are being tracked 
    using their phones. This means that the app will know whi you have been close 
    to. There are however some things which will make this process much harder than 
    it might seem. The first one is that being tracked requires having your phone with you. 
    For many people it is not normal to always carry their phone with them. 
    While this is a small problem that could be alleviated by changing some habits
    the bigger problem is that the corona app is not mandatory in the Netherlands.
    This means that it is very possible that you have been close to people that 
    don't have the app and thus won't be alerted.

    \item \textit{People will know if they have been in contact with a corona patient:}
    This requirements is very similar to the first but is slightly different. 
    The idea of the app is that you will be alerted when you have been close to 
    a corona patient. This requirements has the same problem as the former 
    because it also relies on people using the app. An added problem however is 
    that it is not determined how long you have been close to someone. 

\end{enumerate}

\printbibliography
\end{document}

